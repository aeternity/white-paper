\section{Scaling with state channels}
\label{sect:channels}

The NG technology explained in the previous section improves
scalability by faster preliminary confirmation times for
transactions. It also lifts the restriction of a maxiumum number of transactions in a
generation; when it takes longer to produce the next key-block,
additional micro-blocks can be generated containing new transactions.
This benefits throughput and confirmation times. But if parties really
want confirmation times in milliseconds range and perform more advanced
computations than the total amount of gas in a micro-block would
allow, then state channels come to the rescue.

Conceptually one can regard a state channel as an agreement between
two parties, futher called \textit{participants}, to build a chain of
state changes on the side.
Only the participants have access to this chain in the so called
\textit{state channel}. The parties agree on the initial
state of this chain. Typically, the state would
contain some initial accounts for the involved parties in which they
reserve some tokens from the \blockchain\ for transactions in
the channel. Then they post a joinedly signed transaction on-chain to
confirm that this is the initial state.
The initial amounts are then reserved on-chain and cannot
be used on-chain until they are released.

Apart from initial amounts,
the participants also post the hash of the channel state-tree: the state the
parties do agree upon. By only submitting the hash, the actual state
is kept private to the channel, both parties know, but they need not
reveal it.

If only accounts are used in a state channel, one can consider it a
payment channel, a simpler purpose-specific form of a state channel.
However, one can also agree upon a contract in the
channel, either initially, or added later. In this way, one can also
perform contract calls to update the state of the channel.

The state representation is determined by the state channel
implementation and different implementations may in theory use
their own representation. The state hash is what is visible on-chain
and what parties agree upon in their transactions.
In case of dispute resolution, the
latest agreed, mutually signed, state hash can be posted to the \blockchain.

\subsection{On-chain and off-chain transactions}

In a state channel users communicate peer-to-peer with
each other, using mutually signed transactions that include state
hashes. In this way parties agree upon a state, without any
mining involved. Instead of creating key or micro-blocks, the transactions are just
directly applied to update the state. Only states
jointly agreed upon are recorded. This also
means that there are no transaction or gas costs involved\footnote{In a state
channel participants can agree to have a kind of transaction cost, but it
is not the default setup} and most notably that a transaction
is confirmed as quickly as both parties can sign it. This confirmation
time can therewith be reduced to milliseconds.

A typical use-case for a state channel, or rather a payment channel, is a subscription model, for
example a coffee ``account'' with 25 cups of coffee. Each time one
orders a coffee, the coffee shop
creates a transaction to update the state and both shop
and customer sign this transaction. They agree upon the new state.
Since it would be annoying to have to wait 3 minutes or even 3 seconds
on the payment of a cup of coffee, it is a good example for off-chain
processing. At the same time, it also addresses privacy concerns,
since only the involved parties and not all the users of the
blockchain need to monitor the customers coffee consumption.

Scaling is achieved by performing transactions \textit{off-chain},
i.e.\ transaction in the channel. One starts with an \textit{on-chain}
transaction agreeing and verifying a starting state and balance. After
that many transaction can be performed without involving the
\blockchain. A coffee shop could have thousands of customers, all buying coffee and none of
this puts transaction pressure on the blockchain. Only after
terminating or topping up a subscription, a new on-chain
transaction should be recorded. Having a thousand cups of coffee paid
per second is no technical limitation, but possibly hard to achieve as
a business.

The coffee subscription only includes two accounts and off-chain
payments from one of these accounts to the other.
Notably one
can create contracts in a channel or refer in a channel to contracts
created on the blockchain. We discuss contracts in Sect.\
\ref{sect:sophia}, but one use case is to implement a game-playing
contract (e.g.\ tic-tac-toe) and refer to that contract when playing the
game in a channel. In such use-cases it is also an advantage to be
able to play quickly and not having to wait 3 seconds before a participant's move
is included in a transaction.

\subsection{Disputes}

A state channel requires both parties to sign each state to make
sure the state is agreed upon. Typically a state channel can be closed
under mutual agreement and then a closing transaction is used to
re-distribute and return the reserved balances to the on-chain
accounts. Another way to extract reserved funds from a state channel
is to mutually agree upon a withdrawal from the channel to an account of
one of the parties. Typically, topping up a subscription would be done
in combination with the coffee shop withdrawing part of the funds in
the channel.

But there might be disputes. For example, a customer
could decide not to cancel a subscription, but keep an empty account
in the state channel forever. This is disadvantageous for the shop,
because it cannot extract the funds from the already paid coffee in
mutual agreement. For this reason, there is a solo close transaction,
a way for one party to close the channel on-chain and use the last
signed and agreed state as a proof on how the funds should be divided.

Clearly, there is plethora of scenarios in which one can try to
cheat. One could buy a coffee off-chain and at the same time solo
close the channel on-chain. That would mean a free coffee. Therefore,
funds are not immediately returned after a solo close, but kept for a
certain period, called a \textit{lock period}. During this period the
other party can post a transaction to refute this claim and show a
later state obtained by a mutually signed transaction (the one after
buying the additional coffee). Which then again could possibly be
refuted, etc.

Dealing with disputes is a considerable part of the logic and
implementation of state channels. This becomes even more evident in
the context of using contracts in a channel. A contract may be build
in such a way that it re-distributes balances after a certain state has
been reached. Imagine the above mentioned tic-tac-toe game contract,
where the funds are re-distributed only after that one party has
won. It could then be beneficial for a party to quit the
game when loosing and solo close, or to simply refuse to sign the last
transaction.

Quitting when one expects to lose
harms the other party, because there is no next state that is more
beneficial than the initial state. For this purpose the other party
can then force progress the contract and move to the winning
state. That is, the party can perform a contract call and show that it
ends up in state that can be claimed the actual final state for which
the channel should be closed. This force progress requires more than
just the state hash, here a state has to be revealed that can be used
by the \blockchain\ to execute the next step in the
contract. This requires to folllow a predefined format for the state-trees.

Clearly, a number of different counter
measures are present for the other party in case a malicious force
progress transaction is posted. It is also clear that the state of the
state channel has to be made public (at least partly) to settle a
contract dispute in a force progress transaction.

Users should keep in mind that, by using a state channel, they trade
transaction efficiency and lower transaction fees for some reduction in
safety.
The on-chain transactions are under consensus, but on top of that,
several implementations can be build that offer access to the state
channel with their own implementation logic. For users of a state
channel, it is important to know and understand the limitations of
these off-chain implementations. For example, whether they monitor
on-chain transactions that influence the state channel, such as a
direct solo-close transaction on chain.
Despite, a plurality of dispute handling primitives that are offered to
mitigate the problems dealing with potentially malicious parties,
users may not have the knowledge to use these primitives without a
third party implementation. Therefore it is important to users to understand
whether and how they are implemented.
