\section{Transaction types}
\label{sect:transactions}

The \blockchain\ offers a plurality of different transactions
types, designed to simplify application development for features that are
common or popular in the blockchain sphere. By using smart contracts,
users can develop completely new features themselves, but for already
identified features, a general set of transactions is
provided. These features are spending tokens from one account to
another (Sect.\ \ref{sect:aespend}), naming objects (Sect.\
\ref{sect:aens}), oracles (Sect.\ \ref{sect:aeoracle}), and the
transaction wrappers: generalized
accounts (Sect.\ \ref{sect:ga}), and paying for
(Sect. \ref{sect:payingfor}).
There are also transactions for creating and calling contracts, as
explained in Sect.\ \ref{sect:sophia} and for opening and closing
state channels, deposit and withdraw funds from a state channel as
well as solo closing, slash and force progress to deal with
disputes as mentioned in Sect.\ \ref{sect:channels}.

All transactions are serialized and then cryptographically signed, by default using EdDSA \cite{bernstein2012high}
signatures with elliptic curve Curve25519
\cite{bernstein2006curve25519}. In order to protect against double
spending, each transaction also includes a nonce, which is a stricly
increasing counter connected to the account of the signer.

\subsection{Aeternity accounts}
\label{sect:aespend}

The most basic transaction is a \textbf{spend
transaction} that is used to transfer tokens from one account to
another. An account is identified by the public key of the signer of
a transaction. The spend transaction also serves as the basis for
creating new accounts, by accepting any new recipient public key as a new
account.

Posting a transaction to the chain can fail for many reasons, among
which that the transaction is not correctly signed. But even correctly
signed transactions may fail to become part of the chain.

Each transaction specifies a \textit{fee} that the miner will
     eventually collect when including the transaction in a
     micro-block. This fee can be unattractive, in which case the
     transaction will stay in the transaction pool until it
     expires. Or the fee can be lower than the minimum fee defined by
     consensus, in which case it is an erroneous transaction that
     will be rejected. Including a transaction that is by consensus to be rejected,  such as a
     transaction with lower fee than lowest agreed upon, is called
     fraud. The micro-block is invalid and rejected by
     all other nodes in the network\footnote{Technically posting
       invalid micro-blocks is sometimes possible and a proof-of-fraud
       transaction is posted later to punish the node posting this transaction.}.

Each transaction specifies a \textit{nonce}  to prevent it from replay
attacks \cite{Syverson}. Each new account starts with nonce 1 and as
soon as a transaction with that nonce is accepted on chain, only
transactions with nonce 2 are accepted, etc. A user can post a number
of different transactions with the same nonce in which case it is
non-deterministic which of these transactions will result on chain,
but only one of them will be accepted and the others thereafter
rejected. This can be used as a feature by reposting a
transaction with an unattractive fee with the same nonce and higher
fee. Similarly, a user can post a whole series of transactions with
increasing, but too large nonces. Only when the missing nonce is
posted, all other transactions that possibly remained in transaction
pools are enabled.

\subsection{Aeternity naming system}
\label{sect:aens}

Transferring tokens to a registered name instead of a hard to remember
public key, is a feature that is supported natively by Aeternity.
For example, in a spend
transaction the recipient can be given as a name. For that to work, a
collection of 5 transactions are provided that
resemble internet domain name (DNS) registration.

One of the challenges in name registration is to offer a reasonably
fair system for those that want a specific name. Imagine a user wants
to reserve \textit{emin} as a name. (Since there is currently only one
name space on the \blockchain\ \textit{chain}, technically the
user wants to claim \textit{emin.chain}.)
A short name of size 4, like \textit{emin}, is a name that
possibly many people like to claim. Just posting this name in a
transaction would reveal to everyone monitoring the
transaction pool that the name is attractive to at least someone. It
is rather easy to
immediately post the same name with a higher fee to have the leader
pick that new transaction instead of the already posted request for
the name \textit{emin}. This is called \textit{front running}, getting
your transaction in front of an already posted transaction by paying a
larger fee. Transaction pools are not under consensus, so there are no
guarantees that front running would work, but if there are many
transaction in the system and one does this within the micro-block
creation time of 3 seconds, there is a fair chance that front running
would work.

In order to defend the users against front running, the first step in
reserving a name is the \textbf{preclaim transaction}. In a pre-claim,
one posts the hash of a combination of the name and a random number
(called \textit{salt}). When the preclaim is accepted, you can
post a claim transaction to obtain the actual name.
The \textbf{claim transaction} is
then used to either immediately obtain the name if the name is long
enough (with current governance values, longer than 12 characters), or to start an
auction. In both cases the
claim transaction reveals the name and the salt.

A fee is paid for the
length of the name, shorter names are more expensive and the auction
is open for a certain period expressed in blocks. For the 4 character
name \textit{emin}, the price starts with 134.6269 AE tokens and the
auction would be open 29760 blocks (approx 62 days) after the last valid bid.

At the moment that someone posts a claim,  the name is known. A potential
front running by quickly posting both a preclaim and a claim for the
same name is mitigated by demanding the preclaim and the claim to be
in different generations. In other words, the original claim can be
added to the generation, whereas the new claim needs to wait until the
next generation.

After that the name is claimed, other users can see this name and
claim it with a higher bid. Each bid must be at least 5\% higher than
the previous bid in order to be a valid bid. Note that the bidder must
have enough tokens and that those tokens are reserved in a claim. The
previous bidder gets the tokens returned as soon as a higher bid is
accepted. This means that this users has the funds available for a
possible next bid.

It is very well possible that two users preclaim the same name with
different salt and then claim, but only one of them is accepted, the
other is not. The rejected name claim transaction is not even seen as
the next bid in an auction, even if the price would be higher. There
is a subtle difference between a bidding claim and the original claim
by the \textit{salt} being zero for a bid and non-zero for an original claim.

After the auction, or for long names instantaneously, the highest
bidder owns the name. An additional
\textbf{update transaction} is needed to point the name to something (for example an
account). Additionally there are a \textbf{transfer transaction} to
change the owner of a name, and a \textbf{revoke transaction} to free the
name.

Names have a specific lifetime and there is an agreed maximum number of blocks one can wait
between a pre-claim and a claim in order to succeed with the claim.
Obviously, registered names expire after a while, unless renewed in time
with the update transaction.

When a name has been assigned to an owner and an update transaction
has pointed this name to an account, then one can use the \textit{name
  hash} of the name instead of an account in, for example, a spend
transaction. The name hash function first converts the name to a unique value using the
internationalized domain name standard IDNA \cite{idna2008}. This makes names
case-insensitive and, in particular, it helps to uniquely map
non-ascii characters. After that, the Blake2b hash is applied. This unique
hash is then part of the transaction and the way the name is
internally represented.
IDNA is used, since there are well known attacks by using names that
look similar to the human eye, but are different.

It is important to realize that the names are part of the blockchain
logic. A user should not trust any third party to perform a name lookup on
chain and then substitute the name by an account. If a user want to
transfer tokens to Emin, the user should put the name hash of \textit{emin.chain} in
the transaction and sign this transaction.


\subsection{Aeternity oracles}
\label{sect:aeoracle}


Smart contracts only operate on data that is on the
blockchain. Oracles are a mechanism to bring external data about
real-world state and events onto the blockchain. Data can either be
obtained from large data sources, real-time data, or heavy computations.

Typical external data may be useful for smart contracts. One can base
a decision in a contract on the state of some external data. This can be sensor data or
news events such as stock data, results of a match, supply chain data,
etc.
Researchers try to address the issue of trust in the authenticity of extrenal data
\cite{zhang2016town,guarnizo2019pdfs, adler2018astraea}, but in
general oracles provide data without robust security guarantees.

Oracles are announced to the chain by a \textbf{register oracle
  transaction}. This specifies in what format the oracle expects its
queries ands in what format it is going to respond. Typically this is
specified as a type signature. The register oracle transaction also
includes the fee of the queries to this oracle. Each query must supply
that fee in order to be answered. The query fee is the economic
incentive for the oracle to provide information.

When an oracle is registered on chain, any user can post a
\textbf{query oracle transaction} with the rightly formatted query.
The oracle supplier monitors the blockchain and will see this
query and post an \textbf{oracle response transaction} with the answer
to the query in the predefined format. In this way,
the data becomes part of the data on the
blockchain. This data can be referred to in a smart contract.

\subsubsection{Data as a service}

External data may come from a large database, possibly also accessible in different
ways, but via the oracle made accessible on the blockchain. Typically
one could think of supply chain data. If supply chain data is
accessible via a trusted oracle, one could post an oracle query for
last transaction on a specific item one ordered. Although the answer
on such a query may be interesting and valuable in itself, the main
purpose of asking for it would be to use it in a contract to transfer
some tokens (goods have arrived in harbour, 20\%  of tokens are
transferred).

The above supply chain data may be anonymous enough to appear on a
blockchain. There is, however a privacy issue, external data that is
put on chain is made public. So, even if there might be an interesting
use case, one must be careful with for example personal data. If one
would have an airline oracle that given a last name and booking
reference returns flight data ``date'', ``from'' and ``destination''
airport, then this becomes public data. Having a contract pay the
travel agent when the oracle returns that the correct date and flight has
been booked, is therefore a bad idea. Even encrypting or decrypting the
data in the contract would be a bad idea, since contract state and
operations are visible.

Moreover, one cannot get paid for the same data twice, because the
first time it is posted, it becomes public. Therefore, typical data
normally is rather anonymous or invaluable to others than involved
parties, or is already/will become public, such as the weather or the
outcome of a match. Point is that one can use data that becomes
available in the future to base contractual decisions upon.

\subsubsection{Off-chain computations}

Oracles can also be used to perform heavy computations off-chain and
then post the actual result on-chain. After all, Sophia and the amount
of gas available would make it impossible to implement a chess
calculator to propose the next good move. Implementing this as an
oracle would work. One queries for a certain position and gets a best
move response. Clearly, the chess hints are already freely available, which harms
this business model more than the fact that hints for specific
positions becomes public data.

\subsubsection{Timing}

Users that post a query would normally want a response rather
quickly. Therefore, they can specify query TTL, either absolute
or relative key-block heights. A relative query TTL of 2 assures
that if the oracle does not answer within 2 key-blocks after the query
is accepted onchain, the query fee is not paid. In fact, an answer that is too
late, will not make it on chain and no contract can use it in a
decision.

Oracles have a specific lifetime, supplied in block height when
registering the oracle. After that block height, queries to the oracle
are no longer resulting in a response. The lifetime of an oracle can
be extended using an \textbf{extend oracle transaction}.


\subsubsection{A lottery example}

An example of the use of oracles and contracts can be illustrated with
a little lottery example. Note that all computation on a blockchain
should be deterministic in order to be able to validate the
results. If not exactly the same, the corresponding state hashes will
differ. As a consequence there is no random number generator in the
Sophia language\footnote{Even if some kind of random function would be
  offered, it would be deterministic and hence have a predictable
  outcome}.

Running a simple lottery in which users buy a ticket and after a while
one draws one of the tickets as the winner, is somehow depending on
some kind of fair randomness. If the number is known or computable at
the start of the buying process, one might be able to figure out what
ticket to buy to win the lottery. But if we close the lottery and then
ask an oracle for a random number, then a trusted, but disconnected
computation can be used to draw the winner.

So assume there is such an oracle, monitoring the chain and registered
with a reasonable query fee covering for its cost of operation. The
oracle has an identifier, for the example say
\begin{quote}
  \textit{"ok\_shEHMV8Q2F1HR86pcyF7DYpudg8hnvJwJuVE3berWpbktnL2R"}.
\end{quote}
We can now write a Sophia contract that takes this oracle identifier
as input of its initialization and uses it for random numbers in the
lottery game.

\begin{verbatim}
include "List.aes"

contract Lottery =

  record state = { participants : list(address),
                   price_sum : int(),
                   close_height : int(),
                   oracle : oracle(int, int),
                   query : option(oracle_query(int, int))  }

  entrypoint init(rand : oracle(int, int)) : state =
    { participants = [],
      price_sum = 0,
      close_height = 0,
      oracle = rand,
      query = None }
\end{verbatim}

The contract stores a list of participants in its state, a price sum
that increases for each ticket bought, a closing height after which no
new tickets can be purchased. The initial closing height is zero,
because initially there is no ongoing lottery. The state also contains
an oracle and an optional query. This query will be instantiated when
the lottery is closed and a ticket is drawn.

The creator of this contract may now start the lottery by supplying a relative closing height,
for example, 20 key-blocks from that the transaction gets on chain;
approximately one hour. Any user can then buy a ticket.
\begin{verbatim}
  stateful payable entrypoint start(n : int) =
    require( Call.caller == Contract.creator, "not creator" )
    require( state.price_sum == 0, "lottery ongoing" )
    require( n > 1, "block in future")
    put(state{ participants = [],
               close_height = Chain.block_height + n })

  stateful payable entrypoint buy() =
    require( state.close_height > Chain.block_height, "lottery closed" )
    require( Call.value == 10, "price ticket 10" )
    put(state{ participants = Call.caller :: state.participants,
               price_sum = state.price_sum + 8 })  // we take 20%
\end{verbatim}
Note that a lot of conditions are checked, resulting in abortion of
the contract when falsified. In general, making the contract safe
requires thinking through a large number of possible scenarios in
which things may not work out as expected.
For example, if no users buy a ticket\footnote{The price of the ticket
is set to $10$ aettos instead of a more realistic  $10\_000\_000\_000\_000\_000$ to
make the code more readable}, the contract creator must be
able to restart the lottery at a later time, but the creator should
not be able to restart as soon as there are participants. Similarly,
one should not allow participants to buy tickets when the lottery is
closed.

The keyword \verb+payable+ expresses that we expect the participants
to add a token amount to the contract call transaction, which is
checked by comparing \verb+Call.value+. Similarly, the contract
creator needs to pay something into the contract to cover the oracle query
fee\footnote{Note that there is no check in the contract that it has
  enough funds to query the oracle after starting a lottery. The
  participants are at risk here} in case there are too few
participants.

When the closing height is reached, someone, most likely the creator
of the contract, asks the oracle to draw a number between 1 and the
number of participants. This call returns the query., such that anyone
can monitor the chain to see if the query has arrived. When that's the
case, the winner, or anyone else, can call the claim function, which
will transfer the price sum to the winner.
\begin{verbatim}
 stateful entrypoint draw() : oracle_query(int, int) =
     require( Chain.block_height > state.close_height, "ongoing lottery" )
     require( state.price_sum > 0, "no ongoing lottery" )
     require( state.query == None, "already drawn" )
     let q =
        Oracle.query(state.oracle, List.length(state.participants) - 1,
                     Oracle.query_fee(state.oracle),
                     RelativeTTL(5), RelativeTTL(480))
     put(state{query = Some(q)})
     q

  stateful entrypoint claim() : option(address) =
    switch(state.query)
        None => abort( "no drawing" )
        Some(query) =>
          switch( Oracle.get_answer(state.oracle, query))
             None => abort("waiting for query")
             Some(winner) =>
               let winner_account = List.get(winner, state.participants)
               // Spend to winner
               Chain.spend(winner_account, state.price_sum)
               put(state{ price_sum = 0, query = None })
               Some(winner_account)
\end{verbatim}

The reason to make it possible for anyone to call these functions is
to ensure that the contract creator can force progress to start a new
lottery and the winner to be able to claim even if the contract owner
is not around. The TTLs in the query assure that the oracle has
approximately 15 minutes to answer, long enough to even get it out
in busy times. Within a day one then has to claim the price sum,
otherwise, because the query answer is then no longer on chain.

This contract is far from fully secure, but illustrates an example of
how oracles and contracts can be used together. An easier way to
get a reasonable random number is to use the hash of the key-block at
closing height.




\subsection{Generalized accounts}
\label{sect:ga}

Generalized accounts are a way to provide more flexibility to signing
transactions. Both the nonce handling and the signature checking are
done by a smart contract that is attached to the account. This can,
for example, be useful when one would allow users to sign transactions with
other cryptographic primitives than the default\footnote{Some hardware
  devices may be restricted to other cryptographic signing algorithms
  than the default on the \blockchain.} EdDSA as mentioned in
Sect.\ \ref{sect:transactions}.

If a user wants to have a generalized account, then this user must
provide a smart contract in a \textbf{attach transaction}. This
contract is thereafter attached to the given account. The contract
must have an authentication function that returns a boolean whether or
not authentication is successful. The attach transaction itself is
just like all previously mentioned transactions signed in the default
way. It turns a normal account into a generalized account, and
\textit{there is no way back}.

When an account is a generalized account, any transaction can
be wrapped in a so called \textbf{meta transaction}. That is, one
prepares an ordinary transaction in the usual way, but with a nonce set to
zero. After that, one adds additional fee, gas and authentication data to
run the smart contract. When this transaction is processed, the
authentication function in the smart contract associated with the
account is called with the provided authentication data as input. If
the authentication fails the transaction is discarded, otherwise its
inner transaction is processed.

The following smart contract is an example that allows signing with
the ECDSA algorithm \cite{johnson2001elliptic} and the popular elliptic curve
Secp256k1, used for example by Bitcoin and Ethereum
\cite{bos2014elliptic, mayer2016ecdsa}.

\begin{verbatim}
contract ECDSAAuth =
  record state = { nonce : int, owner : bytes(20) }

  entrypoint init(owner' : bytes(20)) = { nonce = 1, owner = owner' }

  stateful entrypoint authorize(n : int, s : bytes(65)) : bool =
    require(n >= state.nonce, "Nonce too low")
    require(n =< state.nonce, "Nonce too high")
    put(state{ nonce = n + 1 })
    switch(Auth.tx_hash)
      None          => abort("Not in Auth context")
      Some(tx_hash) =>
        Crypto.ecverify_secp256k1(to_sign(tx_hash, n), state.owner, s)

  function to_sign(h : hash, n : int) : hash =
    Crypto.blake2b((h, n))

\end{verbatim}

The contract is initialized by providing the public key used for
signing and the nonce (in the contract state) is set to 1. The
authentication function takes two parameters, the nonce and the signature.
The authorization function checks that the nonce is correct, and then
proceeds to fetch the TX hash from the contract environment using
\verb+Auth.tx_hash+. In this example the signature is for the Blake2b
hash of the tuple of the transaction hash and the nonce). The
authorization finally checks that the private key used for signing the
hash was from the owner.


By attaching this contract to an aeternity account, users can sign
aeternity transactions with their bitcoin private key. They need to
keep track, of course, what nonces they have used for this contract,
to provide the right next nonce.

\subsubsection{Security considerations}

Before the authentication is performed, there is no account that one
can charge for the computational effort of running the authentication
function. After all, anyone could wrap a transaction in a \textbf{meta
  transaction} and submit it. It would be an easy attack to empty a
generalized account if the account had to pay for failed authorization
attempts. So, the gas for authentication is only charged when
successful. This opens up for another unpleasant attack.

Since there is no cost involved for the user to run an authentication
function, but the miner needs to spend execution cycles, one could
potentially write a complex function as authentication function and
extract resources from a miner by calling one's own authentication
function with failing input data. This is mitigated by not allowing
expensive chain operations in an authentication call. Moreover, miners
are free to implement any sophisticated rules for accepting
transactions in their mining pool, such that they can reject this
behaviour when observed.

Using different signature algorithms is only one of many possible uses
of generalized accounts. Other uses cases can be multi-sig, spending
limits (per week/month), limiting the transaction types, and more. For
these applications smart contracts have to be written. Utmost care
needs to be taken when implementing the authorization
function in these smart contracts. If the contract does not enforce
integrity checking or replay protection, then it will be vulnerable to
abuse.


\subsection{Financing transaction costs}
\label{sect:payingfor}

In order to make users enthusiastic about a blockchain application,
one may want them to try it for free. However, there are always costs
involved for transaction and gas. This means that a new user has to
buy tokens at some exchange to pay for the fees. This can be considered
a hurdle for adoption. Of course, one can ask a user for an account and put
some tokens on it, but then those tokens can be used for anything.
The {\AE}eternity solution is more powerful and can be used to pay for
just specific transactions. It can be used to pay for both transaction
fee and gas cost of a contract call.

Assume a game played via a contract on the blockchain. One interacts
with the game, by calling the contract. In order to get more users for
the game, the game provider could make an App that visualizes
the game and asks for a next move. This App could automatically create
an aeternity account, even without the user being aware of it. This
account can be used to sign transactions on the blockchain, but there
are no tokens in the account. This move is then encoded in a
transaction signed by the players account in the App. The transaction
is submitted to the game provider, who inspects it to see that this
is indeed a move in the game and wraps it in a \textbf{payingfor
  transaction} signed by the game provider. The gas and fee are now
paid by the game provider's account.
Clearly this is also a way to have some trustful cross-chain
activity. The App user could provide the game provider with funds on a
different blockchain.

One can pay for any transaction apart from the payingfor transaction
itself. So even a generalized accounts meta transaction can be paid
for, as long as it recursively does not contain any other payingfor
transaction.
