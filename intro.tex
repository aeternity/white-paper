\section{Introduction}

Blockchain technology has been attracting a lot of attention since the
first blockchain was proposed by Nakamoto~\cite{SN} in 2008.
Many blockchain platforms are being developed to serve the next
generation of applications. The novelty of the field provides an
interplay between developers offering new features, enabling new
applications to be written by application developers, and application
developers therewith requiring additional features or possibilities.

Aeternity~\cite{AE,UlfWigerCodeMesh2018} is an open source blockchain
platform aiming at being a development platform for
advanced blockchain applications that can be used by millions of
users. Several key technologies are put in place to meet these scaling
requirements, most notably the state channels, the next generation of
Nakamoto consensus algorithm and the very efficient FATE virtual
machine for smart contract execution.

The plaform runs as a decentralized trustless distributed
ledger with Proof-of-Work (PoW) protected consensus \cite{Tromp2015CuckooCA}.
In Sect.~\ref{sect:mining} we explain how we improve scalability of traditional
block mining by combining it with the Bitcoin NG technology
\cite{Eyal:2016:BSB:2930611.2930615}. The result of this change
is a throughput of about 100 on-chain transactions per second.

Further scaling, towards thousands of transactions per second, is
realized by state channels Sect.\ \ref{sect:channels}, an off-chain encrypted
peer-to-peer communication protocol. After agreeing on-chain to
collaborate in a state channel, parties communicate mutually signed
transactions to each other. Closing the channel, with or without
dispute resolution, is performed on-chain again.

The \blockchain\ offers a variety of different transactions originating from commonly
used applications on other blockchain implementations. For example, by
identifying that some chains implement a way to claim a name for
something on a chain, the \blockchain\ provides a set of
transactions that make this easy for developers, without the need to
implement a smart contract for it. Another example is a set of transactions to register and
query oracles. These transactions are explained in more detail in
Sect.\ \ref{sect:transactions}.

Many features are yet to be invented, but can already be implemented
by users if they use the Aeternity smart contract language
\textit{Sophia}. Sophia is a Turing complete functional language
designed with security in mind. Many mistakes that one can make in
other contract languages are impossible to make or are easier to detect
when using Sophia. In Sect.\ \ref{sect:sophia} we present some key
ideas of the language.

Contracts are compiled to bytecode, which is executed on a highly
efficient virtual machine \textit{FATE}. Similar to other smart contract
languages every operation has a gas cost associated to it. This cost reflects
the amount of work needed to execute a contract. The FATE virtual machine is specifically designed for
Aeternity to meet the high security and efficiency demands, which we
explain in more detail in Sect.~\ref{sect:fate}.

The Aeternity protocol defines the \blockchain, such that
different implementations agree. At the moment there is one
implementation using the functional language Erlang
\cite{Armstrong:2010:ERL:1810891.1810910}, This language originates from the
telecommunication industry and used in large distributed and
concurrent systems (e.g.\ WhatsApp). However, the choice
of implementation language has no further
implications for the techniques used and described in this paper.
